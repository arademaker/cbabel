\documentclass[10pt]{article}% ===> this file was generated automatically by noweave --- better not edit it
\usepackage[latin1]{inputenc}
\usepackage[brazil]{babel}
\usepackage{graphicx}
\usepackage{subfigure,wrapfig}
\usepackage{amsmath,amssymb,amsfonts}
\usepackage{noweb}
\noweboptions{smallcode,longchunks,nomargintag}

\newcommand{\cb}{CBabel}
\newcommand{\cbt}{Maude CBabel tool}
\newcommand{\szt}{Sztajnberg}
\newcommand{\fm}{Full Maude}
\newcommand{\tr}{teoria de reescrita}
\newcommand{\Tr}{Teoria de reescrita}
\newcommand{\rwl}{l�gica de reescrita}
\newcommand{\Rwl}{L�gica de reescrita}
\newcommand{\rrio}{CR--RIO}
%\newcommand{\code}[1]{{\footnotesize{\tt #1}}}
\newcommand{\ingles}[1]{``#1''}
\newcommand{\CBCONF}{\code{CBABEL-CONFIGURATION}}

\begin{document}
\pagestyle{noweb}

\title{A aplica��o produtores e consumidores}
\author{Alexandre Rademaker}
\maketitle

\section{A aplica��o}

O problema dos produtores e consumidores � comumente utilizado na
literatura, por exemplo em \cite{Tanenbaum1987}, para descri��o de
aspectos de coordena��o entre processos concorrentes que disputam
acesso a um recurso comum. Esta aplica��o foi tamb�m extensamente
utilizada em \cite{sztTese} para apresenta��o das primitivas da
linguagem \cb{}.

Nesta aplica��o, um recurso denominado \ingles{buffer} � disputado
entre processos produtores e consumidores. Os produtores acessam o
\ingles{buffer} para depositar itens que eles acabaram de produzir. Os
consumidores acessam o \ingles{buffer} para remover itens que eles
consumir�o. O \ingles{buffer} pode ser limitado ou ilimitado. Quando
limitado, a aplica��o deve garantir que n�o ser�o depositados mais
itens que o limite m�ximo permitido, e n�o ser�o removidos itens
quando o \ingles{buffer} estiver vazio. Nesta se��o, estaremos
implementando e analisando tr�s diferentes arquiteturas para o
problema dos produtores e consumidores. Estas arquiteturas foram
inspiradas nas arquiteturas propostas por \szt{} em \cite{sztTese}.

\nwfilename{pcb-sync.nw}\nwbegincode{1}\sublabel{NW5RLeB-1XPMUq-1}\nwmargintag{{\nwtagstyle{}\subpageref{NW5RLeB-1XPMUq-1}}}\moddef{modules.maude~{\nwtagstyle{}\subpageref{NW5RLeB-1XPMUq-1}}}\endmoddef\let\nwnotused=\nwoutput{}\nwstartdeflinemarkup\nwprevnextdefs{\relax}{NW5RLeB-1XPMUq-2}\nwenddeflinemarkup

module PRODUCER \{ 
 in port put ;
\}

\nwalsodefined{\\{NW5RLeB-1XPMUq-2}}\nwnotused{modules.maude}\nwendcode{}\nwbegindocs{2}\nwdocspar

O modulo consumidor � expresso por:

\nwenddocs{}\nwbegincode{3}\sublabel{NW5RLeB-1XPMUq-2}\nwmargintag{{\nwtagstyle{}\subpageref{NW5RLeB-1XPMUq-2}}}\moddef{modules.maude~{\nwtagstyle{}\subpageref{NW5RLeB-1XPMUq-1}}}\plusendmoddef\let\nwnotused=\nwoutput{}\nwstartdeflinemarkup\nwprevnextdefs{NW5RLeB-1XPMUq-1}{\relax}\nwenddeflinemarkup

module CONSUMER \{ 
 in port get ;
\}

\nwendcode{}

\nwixlogsorted{c}{{modules.maude}{NW5RLeB-1XPMUq-1}{\nwixd{NW5RLeB-1XPMUq-1}\nwixd{NW5RLeB-1XPMUq-2}}}%
\nwbegindocs{4}\nwdocspar

E finalmente podemos definir o modulo buffer.


\end{document}
\nwenddocs{}
